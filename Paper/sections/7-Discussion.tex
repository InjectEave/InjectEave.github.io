\section{Discussion}



\subsection{Feasibility and limitation of analog input devices}
\label{sec:discussion on analog input devices}
%由landline引出
Our investigation into analog input devices was initially prompted by a discovery during the closed-loop attack on the landline phone in \cref{sec:casestudy3}. While our primary objective was to eavesdrop on the remote caller’s voice via the landline’s speaker output, we also eavesdropped on vague and blurred audio during the local user’s speech. Motivated by this finding, we sought to systematically characterize the vulnerability of dedicated analog input devices. We evaluated two representative commercial microphones: the UGreen CM769 and the Razer SEIREN V3 MINI. However, Unlike the 6 meter range achieved with headphones, our tests on microphones revealed a significantly constrained attack distance. Even when the target microphones were capturing audio at maximum input volume, the maximum distance for recovering intelligible audio was limited to approximately 30 cm. 

The fundamental reason of this disparity is the significant amplitude difference in secret signals between input and output devices. While headphone drivers require typically 1–2 V signal to operate, microphones function as passive transducers, generating extremely weak 1–10 mV signal. According to \cref{eq:Vmod}, this 40–60 dB voltage deficit naturally results in a much weaker leakage signal for microphones, causing the secret information to be easily buried beneath the noise floor.

This boundary is determined by the capabilities of the experimental equipment. To quantify this potential, we simplify the propagation model to free-space conditions. Building upon the leakage model in \cref{eq:effi}, we formulate the theoretical maximum eavesdropping distance $d_{max}$ under free-space conditions as:

\begin{equation}
d_{max} \propto \sqrt{\frac{|V_{inj}| \cdot |V_{sec}|}{N_{sys}}}
\label{eq:dist_limit}
\end{equation}

where $N_{sys}$ denotes the aggregate system noise floor, governed by the receiver's sensitivity and the signal generator's phase noise. This formulation reveals that the range constraint can be compensated by optimizing the adversary's hardware. By utilizing high-gain antennas or power amplifier to boost $|V_{inj}|$ and low noise signal generator and spectrum analyzer to suppress $N_{sys}$, adversaries can maximize the injection-to-noise ratio, thereby extending the effective eavesdropping distance $d_{max}$ of input devices.
%理论提升


% \subsubsection{Phase Noise and Equipment Limitations \hr{Too technical, put it in the appendix?}}The recovery of such weak sidebands is further complicated by the phase noise of the RF equipment. Since audio frequencies are located very close to the carrier (e.g., offsets of 300 Hz - 3 kHz), the leakage sidebands fall within the "skirt" of the carrier's phase noise profile. In our setup, the phase noise of the USRP B210 transmitter and the Siglent spectrum analyzer limits the dynamic range at small frequency offsets. When $V_{secret}$ is in the millivolt range, the secret signal are easily masked by the phase noise of the injected carrier itself. Therefore, the range of microphone eavesdropping is currently limited not just by propagation loss, but by the spectral purity of the adversary's equipment. Utilizing a high-end signal generator with ultra-low phase noise and a receiver with a lower noise figure could theoretically improve the SNR, potentially extending the range beyond the current 30 cm limit. 

% \subsection{Limitations and Attack Scope}
% \textbf{Stealthiness and Spectral Detectability.} Our current implementation prioritizes modulation efficiency over stealth by transmitting a Continuous Wave (CW) carrier at relatively to induce non-linearity. This creates a distinct peak in the frequency spectrum, rendering the attack susceptible to detection by standard spectrum monitoring systems. To mitigate this, future research may explore Spread Spectrum (SS) techniques. By modulating the injection carrier with a pseudo-random code, the injection energy can be spread across a wider bandwidth, effectively lowering the Power Spectral Density (PSD) below the environmental noise floor while allowing leakage recovery via processing gain.

% \textbf{Focus on output device.} Furthermore, our evaluation primarily demonstrates long-range success on analog output devices, such as headphones and smart home actuators. While we confirmed vulnerability in analog input devices such as microphones, the effective attack range was significantly restricted. This disparity is not an implementation oversight but stems from fundamental physical differences in signal intensity between device types, which we characterize in the following section. \ly{todo: mention: mic of landline phones here; also, motivate the next discussion on inputs}


\subsection{Mitigation}

Mitigating \alias{} is a grand challenge, as it combines adversary-controlled EM injection to overcome conventional EM side channel protections. To address this, we evaluate three categories of defense strategies: EM hardening, conventional side-channel protections, and injection-specific countermeasures.

\textbf{Traditional EM Hardening.}
%shielding filter Cryptographic randomnization 
The first line of defense typically involves electromagnetic compatibility (EMC) hardening techniques, such as shielding, filtering\cite{kune2013ghost}, and twisted-pair cabling\cite{adi_mt095_emi}. While these methods reduce coupling efficiency, they do not eliminate the vulnerability. As shown in \cref{eq:dist_limit}, the eavesdropping range scales with injection voltage, meaning an adversary can theoretically overcome shielding simply by increasing transmission power to compensate for the attenuation. Furthermore, standard filters often degrade at high frequencies due to parasitic inductance \cite{szakaly2024assault}, rendering them ineffective without the use of costly specialized components.

\textbf{Passive EM side-channel defenses.}
Conventional defenses developed for passive electromagnetic side-channel attacks, such as cryptographic masking\cite{camurati2018screaming}, or randomized clocking \cite{onishi2025sound}, also prove largely ineffective against this threat model. These strategies are specifically designed to obfuscate digital logic transitions and protect cryptographic states . In contrast, Injection-Induced EM Side Channels target the analog signals. Unlike digital data, these analog signals cannot be mathematically masked or encrypted without irreversibly degrading the signal fidelity.

\textbf{Active injection counter measures.}
Given the limitations of physical hardening and the inapplicability of digital obfuscation, the most viable defense strategy involves detecting and mitigating the active injection source itself. Promising countermeasures include environmental monitoring \cite{adami2011hpm,adami2014hpm}, signal contamination detection\cite{xu2021inaudible,gao2023practical}, and sensor fusion\cite{tu2021transduction,jin2024phantomlidar}. Environmental monitoring employs dedicated antennas to continuously scan the ambient spectrum for high-power RF carriers indicative of an attack , while signal contamination detection leverages redundant circuits to identify anomalous high-frequency components or DC offsets induced by rectification. Additionally, sensor fusion techniques can utilize software-level consistency checks to correlate sensor inputs with expected system states. However, these injection-specific defenses incur significant implementation overheads, including increased hardware costs and power consumption for monitoring modules. Furthermore, many detection mechanisms rely on complex system modeling that may yield false positives in noisy environments and lack validation against the sophisticated closed-loop manipulation capabilities demonstrated in this work.

\section{Related Work}

\subsection{EM Injection and Side-channel Leakage}  
Electromagnetic (EM) injection and EM side-channel leakage represent two complementary electromagnetic attack paradigms.
In EM injection attacks, an adversary \textit{actively transmits carefully crafted EM signals} to manipulate the sensing, computation, or actuation of a victim device, thus violating its integrity or availability. EM injection has been extensively demonstrated as a powerful attack vector against a broad range of cyber-physical systems (CPSs) to induce faults~\cite{hayashi2012transient,hayashi2014precisely,menu2020experimental}, manipulate sensor output~\cite{tu2019trick,kune2013ghost,zhang2024virtual}, or disrupt system execution~\cite{yang2024rethink,jiang2023glitchhiker,shan2022invisible}. Prior work shows that EM injection attacks can interfere with sensing, computation, actuation, and data communication of CPSs across diverse application domains, including critical infrastructure~\cite{tu2019trick,yang2024rethink}, autonomous driving~\cite{jiang2023glitchhiker,jin2024phantomlidar,ren2025ghostshot}, medical healthcare~\cite{kune2013ghost,hossen2023first,armengol2023brain}, IoT devices~\cite{zhang2024virtual,shan2022invisible,wang2022ghosttouch,jiang2024ghosttype,yang2025lightantenna} and cryptographic modules~\cite{hayashi2012transient,hayashi2014precisely,nishiyama2023remote,menu2020experimental}.

In contrast, EM side-channel leakage exploits \textit{passively received unintended EM emanations} from a victim device to recovering the cryptographic key of a target device~\cite{gnad2019leaky,camurati2018screaming,genkin2015stealing} or infer sensitive user-specific data, compromising data confidentiality. Early and seminal work in this area is exemplified by TEMPEST attacks, where EM leakage can be exploited to reconstruct visual information displayed on computer screens~\cite{van1985electromagnetic,kuhn2002optical,kuhn2005electromagnetic}. Building on this foundation, subsequent research has demonstrated that EM side-channel leakage  can be used to recover a wide range of sensitive user-specific and secret data from consumer devices, including audio~\cite{choi2020tempest,chen2024eavesdropping,onishi2025sound}, keystrokes~\cite{jin2021periscope}, smartphone displays~\cite{liu2020screen}, biometric data~\cite{li2025emiris,ni2023recovering,xu2025empalm}, and even confidential video streams from a smart home camera~\cite{long2024eye}.


Despite their shared underlying mechanism in EM signal transmission and reception , existing research largely treats EM injection and EM side-channel leakage as distinct and independent threat vectors, focusing on fault- and disruption-oriented integrity violations and passive confidential threats, respectively. 
% The extent to which EM injection can actively stimulate side-channel leakage has received limited attention.
In this work, we bridge this gap by theoretically and experimentally showing that EM injections can be used to actively induce side-channel leakage via hardware non-linearity, redefining the boundary between active integrity injection attacks and passive confidentiality attacks and enabling closed-loop control capabilities of victim devices.

% A limitation of EM Injection is the lack of closed-loop control capabilities 
% , thereby transforming EM injection from an integrity-oriented attack into a confidentiality-breaking primitive.

\subsection{Wireless Sensing and Eavesdropping}
Wireless sensing and eavesdropping techniques transmit mmWave~\cite{wang2022mmphone,wang2022mmeve,hu2022milliear,basak2022mmspy,li2020wavespy,xu2019waveear}, acoustic~\cite{cheng2020sonarsnoop,halevi2015keyboard,roy2016listening}, or optical~\cite{sami2020spying,luo2025laserkey,nassi2023little} probing signals, and analyze the resulting reflections to infer information about a target system. In these approaches, the underlying mechanism is that changes in object state implicitly produce observable variations in the probed signals, enabling inference without direct physical access.  Recent research has also been demonstrated by exploiting backscatter effects and impedance-based analysis. For example, RF-parrot~\cite{yang2024rf} demonstrates audio eavesdropping on wired audio systems by capturing backscattered RF signals from the wire itself; however, it assumes hardware modification of the target device, requiring the integration of a retroreflector to enhance the backscattered signal. 
Pu et al.~\cite{pu2025your} and Shugo et al.~\cite{kaji2023echo} show that digital serial communication data can be eavesdropped through impedance-modulated RF backscatter under active probing without any hardware modifications.
\qh{I am not sure if we shou mention this recent Active EM SCA work~\cite{kitazawa2026active}.} \ly{May want to add the concept and works of impedance side channel} \ly{digital eavesdropping: can be defeated by encryption}

Despite their effectiveness, existing wireless sensing and eavesdropping techniques rely on reflection- or backscatter-based inference driven by impedance variations, where leakage characteristics are limited by the physical properties, impedance profile and environment of the target.
In contrast, our work shows that electromagnetic injection can actively induce side-channel leakage via hardware non-linearity, enabling controllable and closed-loop EM eavesdropping. Crucially, this capability enables EM eavesdropping attack vectors previously considered infeasible, including secret information leakage with very low-frequency components (e.g., on the order of 10~Hz–1~MHz), which are typically too weak to propagate to external eavesdroppers through passive or reflection-based mechanisms. \ly{Need to hoghlight the difference between digital and analog secrets. I can help update later}

% A separate line of works exploits impedance- and backscatter-based analysis
% Mostly talking about the concept of reflection that measures the change of object status, such as mmWave sensing.

% ``Back-scattering'' is another type of ...: Rf-parrot~\cite{yang2024rf}

% Then, also mention the S\&P 2025 work ``Your Cable''~\cite{pu2025your}
% Sound-of-inference~\cite{onishi2025sound}
% Echo TEMPEST~\cite{kaji2023echo}
% Active EM SCA~\cite{kitazawa2026active}
% \subsection{Other Actively Induced Leakage}
% Actively induced leakage has also been explored by acoustic and optical stimulation. 



% \subsection{Backscattering Communication}
% \ly{moved this from sec 2 to here. will revise it}
% % Backscattering is a technique widely used in low-power, or even battery-less devices such as RFID devices [cite] for digital communication. A sender trying to communicate with a target device transmits a carrier signal $V_{carrier}(t)$. The target device modulates a digital message $m = \{0, 1\}$ \ly{I will change the notation of m} onto $V_{carrier}(t)$ by toggling an impedance connected to its antenna between two states, representing 0 and 1, respectively. The different impedance states change $h_{ref}$, the efficiency of the target's antenna that reflects the carrier. The message signal received by the sender can be modeled as 
% % \begin{equation} \label{eq:backscattering}
% % V_{msg}(t) = h_{prop}\{h_{ref}(m, V_{carrier}(t))\},
% % \end{equation}

% % The comparison between Equation~\ref{eq:conventional-em} and Equation~\ref{eq:backscattering} shows a similar structure: when $h_{leak}(V_{sec}(t))$ is modeled as $h_{ref}(m, V_{carrier}(t))$, then backscattering can also be regarded as a form of EM side-channel leakage, except that the secret message $m$ is intentionally modulated onto the carrier and then reflected. 
% Backscattering is a technique widely used in low-power, or even battery-less devices such as RFID devices [cite] for digital communication. An interrogator trying to communicate with a transponder device transmits a carrier signal $V_{carrier}(t)$. The transponder modulates a digital message stream $b[n] \in \{0, 1\}^n$ onto $V_{carrier}(t)$ by toggling an impedance connected to its antenna between two states, representing 0 and 1, respectively. The different impedance states change $h_{ref}$, the efficiency of the target's antenna that reflects the carrier. The message signal received by the interrogator can be modeled as: $V_{msg}(t) = h_{ref}(b[n], V_{carrier}(t))$. 
% % \begin{equation} \label{eq:backscattering}
% % V_{msg}(t) = h_{ref}(b[n], V_{carrier}(t)).
% % \end{equation}


% The comparison between 
% The comparison between Equation~\ref{eq:conventional-em} and \ref{eq:backscattering} shows an interesting observation: the reflected signals in backscattering may be considered as a special form of EM side channel leakage when $V_{sec}(t)$ is set equal to $b[n]$, except that the secret message $b[n]$ in backscattering is intentionally modulated onto the carrier and reflected. 


% Indeed, recent research has demonstrated that a similar reflection phenomenon could be observed in the digital data transmission circuit of computer peripherals~\cite{pu2025your}, where the transmission of 0s and 1s in serial communication interfaces could also unintentionally create two different impedance states, allowing an adversary to eavesdrop on the transmitted digital data in a backscattering-like manner. However, the fundamental modulation process and model of such leakage remain unknown, which limits the types of leakage to binary 0/1 data in \cite{pu2025your} due to the insufficient capability of the coarse-grained impedance toggling model. As we will show in this work, by innovating a nonlinearity-centric theoretical model, it is possible to extend the attack surface from high-voltage digital interfaces to even more pervasive analog interfaces, and to decode fine-grained secrets from continuous signals. 

