\section{Evaluation in a Laboratory Setting}
\label{sec:evaluations_in_lab}

Building upon the analysis of injection-induced EM side channel, in this section, we evaluate the effectiveness of our proposed attack in a controlled laboratory environment. We evaluate \alias{} attack on 11 commercial devices in 5 categories and quantify the attack's performance in real-world settings.


\subsection{Experimental Setup and }
\label{sec:experiment_setup}



\begin{table*}[t]
\centering
\footnotesize
\renewcommand{\arraystretch}{1.3}

\begin{threeparttable}

\caption{Summary of Attacks on COTS Devices}
\label{tab: Attacks on COTS Devices}

\begin{tabular}{|l|l|l|c|c|c|c|c|c|}
\hline

% Header
\textbf{Device Type} & \textbf{Brand} & \textbf{Model} & \textbf{Year} & \textbf{\makecell[c]{Source\\of leaks}} & \textbf{\makecell[c]{Injection\\Frequency}} & \textbf{\makecell[c]{SNR \tnote{\ddag} }} & \textbf{\makecell[c]{Recog.\\Rate \tnote{\S}}} & \textbf{\makecell[c]{Max\\Dist. \tnote{\P}}} \\
\hline

% Wired Headphones
\multirow{3}{*}{\textbf{\makecell[l]{Wired\\Headphones}}}
& Sony + Dell\tnote{*} & ZX110AP & 2014 & \multirow{3}{*}{Amplifier} & 510--570\,MHz & 22.5\,dB & 30/30 & 5\,m \\ \cline{2-4} \cline{6-9}
& Sony + Mac\tnote{*} & ZX110AP & 2014 & & 220--260\,MHz & 14.5\,dB & 30/30 & 4\,m \\ \cline{2-4} \cline{6-9}
& Apple + iPhone\tnote{*}& Earbuds & 2016 & & 1160--1180\,MHz & 6.4\,dB & 29/30 & 1\,m \\
\hline

% Wireless Headphones
\multirow{3}{*}{\textbf{\makecell[l]{Wireless\\Headphones}}}
& UGreen\tnote{\dag} & MAX2 & 2024 & \multirow{3}{*}{Amplifier} & 900--980\,MHz & 23.8\,dB & 30/30 & 6\,m \\ \cline{2-4} \cline{6-9}
& PHILIPS & TAH2020 & 2025 & & 1590--1650\,MHz & 21.6\,dB & 30/30 & 6\,m \\ \cline{2-4} \cline{6-9}
& HP & H231R & 2023 & & 980--1020\,MHz & 20.7\,dB & 29/30 & 4\,m \\
\hline

% Landline
\textbf{\makecell[l]{Landline}} & Flyingvoice & P23GW & 2023 & \makecell[c]{ADC \&\\Amplifier} & 840--920\,MHz & 13.5\,dB & 30/30 & 3\,m \\
\hline

% Fan
\multirow{2}{*}{\textbf{Fan}}
& OIDIRE\tnote{\dag} & ODI-MF10A & 2023 & \multirow{2}{*}{\makecell{Switching\\MOSFET}} & 440--520\,MHz & 39.1\,dB & 30/30 & 6\,m \\ \cline{2-4} \cline{6-9}
& XIAOMI & BPLDS10DM & 2025 & & 530--610\,MHz & 30.5\,dB & 30/30 & 4\,m \\
\hline

% Lamp
\multirow{2}{*}{\textbf{Lamp}}
& JINGZAO & JDO-06 & 2024 & \multirow{2}{*}{\makecell{Power\\Converter}} & 80--650\,MHz & 20.7\,dB & 29/30 & 3\,m \\ \cline{2-4} \cline{6-9}
& XIAOMI\tnote{\dag} & 1S & 2019 & & 70--700\,MHz & 22.4\,dB & 30/30 & 3\,m \\
\hline

\end{tabular}

% 关键修改:添加了 [para] 选项
\begin{tablenotes}[para]
    \footnotesize
    \item [*] Three different host devices for wired headphones.
    \item [\dag] Devices evaluated in \cref{sec: Impact Quantifications}.
    \item [\ddag] SNR is evaluated at 20\,cm distance.
    \item [\S] Recognition rate is evaluated at 20\,cm distance.
    \item [\P] Max distance is evaluated at 5/30 recognition rate.
\end{tablenotes}

\end{threeparttable}
\end{table*}

\cref{fig:experiment setup} shows the experiment setup of the victim’s devices and adversary’s devices.
The victim’s devices comprise a variety of devices ubiquitous in daily environments, serving as representative attack surfaces in real-world daily scenarios. The adversary’s devices are used to inject carefully crafted EM signals to the victim devices to perform injection-induced EM side-channel analysis. Crucially, the entire attack is conducted without physical access, requiring no hardware or software modifications to the victim devices to ensure the results reflect standard consumer deployment.

\textbf{Victim’s devices.} 
We evaluate the real-world performance of \alias{} attack on 11 COTS devices across 5 categories: (1) 3 wired headphones and  (Sony MDR-ZX110AP with MacBook Pro (14-inch, 2023), Sony MDR-ZX110AP with Dell G5, and Apple earbuds with iPhone 15 Pro); (2) 3 wireless headphones (UGreen MAX2, PHILIPS TAH2020, HP H231R); (3) a wireless landline (Flyingvoice); (4) 2 smart fans (OIDIRE ODI-MF10A, XIAOMI BPLDS10DM); and (5) 2 smart lamps (JD JINGZAO JDO-06, XIAOMI 1S). The detailed information of each device is specified in~\cref{tab: Attacks on COTS Devices}, all evaluations are conducted on devices in their original enclosures without any hardware or software modifications.
% We evaluate the real-world performance of \alias{} attack on 13 COTS devices across 5 categories: \textit{audio peripherals} inluding wired headphones (Sony MDR-ZX110AP, JBL JBLT110, Apple Earbuds), wireless headphones (UGreen MAX2, PHILIPS TAH2020, HP H231R), and wireless landline (Flyingvoice), as well as \textit{smart home applications} including fans (OIDIRE ODI-MF10A, XIAOMI BPLDS10DM) and lamps (JD JINGZAO JDO-06, XIAOMI 1S). These categories were selected for their high market penetration and diverse privacy implications: while audio peripherals are direct vectors for eavesdropping, smart home IoT devices like can be exploited to infer user presence and behavior.
% Notably, all devices under test remained in their original enclosures with no hardware or software modifications.
% microphones (UGreen CM769, Razer SEIREN V3 MINI)

\textbf{Adversary’s devices.} The adversary’s devices consist of an EM signal injection module and an EM leakage receiving and decoding module. 
For signal injection, we use an Ettus USRP B210 software defined radio to perform a frequency sweep experiment from 100~MHz to 1~GHz to identify the specific resonant frequencies of the victim device for injection-induced side channel induction. 
To capture and analyze injection-induced EM side-channel leakages, we utilize a Siglent SSX 3075X Plus spectrum analyzer to capture injection-induced signals and conduct subsequent processing to reconstruct the sensitive information.
We report the injection frequencies for each device tested in~\cref{tab: Attacks on COTS Devices}.
% \qh{conduct the frequency sweeping experiment from 70 to 2000~MHz.}
\ly{mostly duplicate to 3.4. Need to trim it}



\begin{figure}[t]
    \centering
    \includegraphics[width=1\linewidth]{figures/5.1/setup1.pdf}
    \caption{Experiment setup}
    \label{fig:experiment setup}
\end{figure}

\textbf{Metrics.} We evaluate the overall performance of the proposed \alias{} attack signal-level, feature-level, and semantic-level perspectives using three primary metrics:
(1) \textit{Signal-to-Noise Ratio (SNR)} is defined to characterize the signal-level quality of the injection-induced EM leakage at specific distances. 
(2) \textit{Signal recognition rate} is defined to quantify the adversary's success in receiving and decoding the injection-induced EM leakage.  It is defined as the proportion of successful recognitions out of 30 independent evaluation trials for each device. Specifically, a trial is deemed successful if the adversary can accurately extract the expected signal features from the captured injection-induced EM leakage—namely, the correct demodulation of a single-tone signal for audio peripherals, or the precise identification of the fan’s rotation speed and the lamp’s 50~Hz power frequency.
(3) \textit{Attack Success Rate (ASR)} provides a unified measure of information fidelity relative to the ground truth. For audio devices, ASR is defined as $1 - WER$, where the word error rate (WER) evaluates the semantic accuracy of recovered speech. For state-driven devices, including smart fans and smart lamps, ASR is defined as the classification accuracy ($N_{\text{correct}} / N_{\text{total}}$) across all operational states.


\subsection{Evaluations on COTS devices}
In this section, we evaluate the \alias{} attack on 11 commercial off-the-shelf (COST) devices. We categorize the analysis into two distinct threat dimensions based on the attack surfaces and exposed privacy risks: high-fidelity audio recovery and human activity inference. Specifically, audio peripherals serve as direct attack vectors for speech eavesdropping, facilitating the recovery of sensitive conversational content. In contrast, smart home devices are leveraged to infer user presence and behavioral patterns by analyzing their unique state-related leakage.


% These categories represent critical attack surfaces with distinct privacy implications: while audio peripherals serve as direct vectors for speech eavesdropping, smart home devices can be exploited to infer user presence and behavioral patterns. Our analysis categorizes these threats into high-fidelity audio recovery and functional signature extraction.

\subsubsection{Attacks on Audio Peripherals}
Audio peripherals, including wireless headphones, wired headphones, and landlines, are ubiquitous in sensitive and private environments such as hotel rooms, conference rooms, and private offices. Despite their proximity to sensitive conversations, these devices often serve as the unprotected terminal stage of the audio processing chain. To evaluate \alias{} attack on these COTS audio peripherals, we play a 2~kHz reference single-tone signal on each device during an injection frequency sweep \ly{I will add a sentence to justify why using 2k Hz}. The successful recovery of this 2~kHz component via a spectrum analyzer confirms the device’s susceptibility to \alias{} attack. As shown in~\cref{tab: Attacks on COTS Devices}, \alias{} achieves a near-100\% signal recognition rate across the wired headphones, wireless headphones and landlines, with each device exhibiting injection-sensitive frequency ranges that allow flexible carrier selection.


\textbf{Wired headphone.}
We evaluate the \alias{} attack on wired headphones to demonstrate effectiveness across different host-device interfaces. As summarized in \cref{tab: Attacks on COTS Devices}, we achieve a near 100\% recognition rate at 20~cm across all setups. For the Sony ZX110AP, the optimal injection frequency shifts significantly depending on the host: it yields a peak SNR of 22.5~dB at 510--570~MHz when connected to the Dell G5, but requires a lower band of 220--260~MHz when paired with the MacBook Pro M2. Despite these variations, the attack remains robust with effective ranges of 5~m and 4~m, respectively. In contrast, the Apple Earbuds on an iPhone 15 Pro exhibit a higher resonant point at 1160--1180~MHz with an SNR of 6.4~dB and a 1~m range
These results indicate that the analog amplifier of host device’s internal sound card is the fundamental component exploited for injection-induced EM leakage, while the headphone cable primarily acts as an unintentional antenna. The distinct grounding and circuit layouts of the Dell and MacBook sound cards dictate the specific resonant frequencies required for inter-modulation, resulting in the same headphone susceptible at different injection frequency bands depending on its host device.

\textbf{Wireless headphone.} 
We evaluate three wireless models, each exhibiting high susceptibility to \alias{} across distinct frequency bands as detailed in~\cref{tab: Attacks on COTS Devices}. The UGreen MAX2 demonstrates the most robust leakage, achieving a SNR of 23.8~dB and a 100\% recognition rate within the 900--980~MHz range, with an effective distance of up to 6~m. Similarly, the PHILIPS TAH2020 supports a 100\% recognition rate within the 1590--1650~MHz range with an SNR of 21.6~dB and a 6~m range. Even the HP H231R maintains a 29/30 recognition rate and a 4~m distance within the 980--1020~MHz band.
We tear down the UGreen MAX2 wireless headphone to further investigate the root cause of these leakages. Our analysis reveals that while the Bluetooth system-on-chip (SoC) is shielded, the 10--15~cm internal analog wires connecting the amplifier and the digital-to-analog converter (DAC) to the dynamic drivers are completely unshielded. These wires effectively act as unintentional antennas to be exploited. 
We further validate the real-world threat of \alias{} in \cref{sec:casestudy1} and \cref{sec:casestudy2}, where we achieve successful through-wall audio eavesdropping in real-world hotel and conference room scenarios.

% We evaluate the attack on three wireless models: UGreen MAX2, PHILIPS TAH2020, and HP H231R.  By sweeping the carrier frequency, we identified optimal injection points for each model, such as 1620~MHz for the PHILIPS TAH2020 and 940~MHz for the UGreen MAX2. Our experiments show that the UGreen MAX2 achieved the highest signal quality with an SNR of 23.8~dB, maintaining a reliable recognition rate at a maximum distance of 6~m. These results confirm that even high-end wireless headphones are susceptible to physical-layer eavesdropping, as the injection-induced side-channel renders upper-layer cryptographic defenses irrelevant.

\begin{figure}[t]
    \centering
    \includegraphics[width= 1.0\linewidth]{figures/5.2/attack_headphone.pdf}
    \caption{Attack on UGreen Max2 wireless headphone.}
    \label{fig:headphone}
\end{figure}


% \qh{Before conducting the in-the-wild attack, we analyzed the physical layout of the target wireless headphones (UGreen Max2) to understand the root cause of the leakage. Teardowns reveal that while the Bluetooth SoC is shielded, the analog output wires connecting the internal DAC/Amplifier to the dynamic drivers are completely unshielded and run along the headband or ear cup circumference. These wires, typically 10-15 cm in length, effectively act as unintentional antennas. Crucially, they carry the demodulated, baseband analog audio after the Bluetooth stack has performed decryption. }


\textbf{Landline.}  To assess risks in professional environments, we extend our evaluation to a Flyingvoice P23GW VoIP landline. We simulate an active call and sweep the carrier frequency to locate sensitive non-linear junctions within the handset's internal amplifier and ADC stages. As summarized in~\cref{tab: Attacks on COTS Devices}, the system exhibits high susceptibility, achieving a 100\% recognition rate and an SNR of 13.5~dB within the 840--920~MHz band. The attack remains viable at distances up to 3~m, allowing an adversary to eavesdrop on confidential corporate negotiations without compromising the digital network. We further demonstrate the severity of this threat in~\cref{sec:casestudy3}, where \alias{} is leveraged to achieve closed-loop manipulation of the landline's voice interface.

\textbf{Effect of Audio Signal Enhancement.}
We further examine the effectiveness of the eavesdropping design in Section~\ref{sec:design} on 2 minutes of speech audio recordings captured at a distance of 50cm with 65dB volume from the UGreen MAX2 wireless headphone using both SNR and Short-Time Objective Intelligibility (STOI), a standard speech quality metric. The results show notable improvements across all STOI and SNR. The average SNR increases drastically from 7.0 dB to 16.1 dB, indicating that the model successfully suppressed the dominant background noise and the injection-induced carrier phase noise. Furthermore, STOI sees a significant boost from 0.58 to 0.72. Given that STOI is highly correlated with human speech intelligibility, this improvement confirms that our algorithm effectively reconstructs the phonetic details masked by the hardware's non-linear harmonics, rendering the eavesdropped speech intelligible to human listeners. 

\subsubsection{Attacks on Smart Home Applications}
Beyond audio-centric devices, we evaluate the generality of the attack on smart appliances that are often considered ``low-security'' but are deeply integrated into private environments. In these devices, the leakage source shifts from audio amplifiers to the Switching MOSFETs in fans or Power Converters in lamps.

\textbf{Smart fans.} We evaluated the OIDIRE ODI-MF10A and XIAOMI BPLDS10DM smart fans. These devices feature operational modes—such as Sleep, Natural, and Standard—that serve as proxies for the user's activity. The attack exploits the modulation caused by the motor's driving signal, where the rotational speed (e.g., low:31~Hz medium:42~Hz high: 52~Hz) modulates the injected carrier. As shown in~\cref{tab: Attacks on COTS Devices}, the OIDIRE model yielded a high SNR of 39.1~dB at 480~MHz, maintaining a successful recognition rate up to 6~m. Unlike audio devices, the fan's leakage manifests as sidebands at distinct frequency offsets in the frequency domain (see \cref{fig:fan}). This enables ``Context Inference'' attacks; for example, detecting ``Sleep Mode'' at night can confirm a user's rest schedule and infer occupancy without the need for visual surveillance.
\qh{need to cite~\cref{fig:fan}.}

\begin{figure}[t]
    \centering
    \includegraphics[width= 1.0\linewidth]{figures/5.2/attack_fan.pdf}
    \caption{Attacks on fan.}
    \label{fig:fan}
\end{figure}

\textbf{Smart lamps.} We further tested the XIAOMI 1S and JD JINGZAO JDO-06 smart lamps. The mechanism exploits the non-linearity of the lamp’s power converter, where the $50\text{ Hz}$ AC mains current modulates the injected carrier (e.g., at $100\text{ MHz}$). As indicated in Table~\ref{tab: Attacks on COTS Devices}, the XIAOMI 1S exhibited strong leakage with an SNR of $22.4\text{ dB}$. Since the sideband amplitude is proportional to the power load, we can remotely infer the precise dimming level. By mapping these power levels to vendor-specific presets (e.g., 20\% brightness for ``Reading''), an adversary can perform ``Behavioral Profiling,'' transforming a simple light source into a beacon that exposes a user's specific activities and routines without requiring any network-level access.
\qh{need to cite~\cref{fig:lamp}.}

\begin{figure}[t]
    \centering
    \includegraphics[width= 1.0\linewidth]{figures/5.2/attack_lamp.pdf}
    \caption{Attacks on Lamp.}
    \label{fig:lamp}
\end{figure}




\subsection{Impact Quantification}
\label{sec: Impact Quantifications}

To characterize the physical limits and practical constraints of the \alias{} attack, we select three representative devices for impact quantification: the UGreen Studio MAX 2 wireless headphones, the OIDIRE ODI-MF10A smart fan, and the Xiaomi 1S smart lamp. Unless otherwise specified, our default experimental configuration employs a 50~cm attack distance, a 90$^\circ$ antenna orientation, and an injection power of 18~dBm. The carrier frequencies are configured at 940~MHz, 480~MHz, and 100~MHz for wireless headphones, smart fan, and smart lamp, respectively, corresponding to their optimal resonant points discovered during the frequency sweep experiment. We quantify the attack robustness by systematically varying the attack distance to determine the effective range, adjusting the antenna orientation to analyze polarization sensitivity, and introducing various material barriers to evaluate signal penetration and eavesdropping in realistic environments.

\textbf{Impact of attack distance.}
To quantify the effective eavesdropping range and the limits imposed by propagation path loss, we varied the distance between the adversary and the victim device up to 5~m. As shown in~\cref{fig:impact_distance}(a), the leakage SNR decreases monotonically as distance increases. Specifically, the SNR for all device categories drops by approximately 45~dB from 10~cm to 5~m.
The attack success rate (ASR) results in~\cref{fig:impact_distance}(b) reveal a significant disparity in hardware-level resilience across different devices. While the UGreen Studio MAX2 wireless headphone's ASR drops to 4.4\% at 5~m and the Xiaomi 1S smart lamp becomes resilient to the attack beyond 2~m, the OIDIRE ODI-MF10A smart fan's persistent 100\% ASR.


Notably, the attack distance can be further extended by using high-performance professional attack equipments. For example, by utilizing a high-performance Keysight N9000B spectrum analyzer, we demonstrate that the effective range at an ASR threshold of 20\% extends to 6~m for the UGreen Studio MAX 2 wireless headphone, 8~m for the OIDIRE ODI-MF10A smart fan, and 3~m for the Xiaomi 1S smart lamp. These results highlight that \alias{} poses a realistic threat to user privacy at practical distances, bypassing the need for close-range physical proximity. 
We provide three case studies in~\cref{sec:case_studies} to show the real-world threat of \alias{} attack in the wild.

\begin{figure}[t]
    \centering
    \includegraphics[width=\linewidth]{figures/5.3/impact_distance.pdf}
    \caption{Impact of attack distance on (a) SNR Trend and (b) Attack Success Rate (ASR) across different COTS devices.}
    \label{fig:impact_distance}
\end{figure}


\textbf{Impact of antenna angle.}
To investigate the impact of the relative angle between the transmitting and receiving antennas on \alias{} attack, we maintain a fixed transmission antenna position while rotate the receiving antenna along the azimuthal plane of the target from $0^\circ$ to $315^\circ$ in a step of $45^\circ$. As shown in~\cref{fig:impact_angle}(a), the leakage exhibits strong directionality due to these anisotropic radiation patterns. All tested devices show prominent peaks at $90^\circ$ with the SNR reaching 18.3~dB for the headphone, 31.1~dB for the smart fan and 14.4~dB for the smart lamp.
The ASR follows these SNR patterns but shows a significant disparity in task-specific robustness. As shown in~\cref{fig:impact_angle}(b), the ASR varies sharply from 50.5\% at $0^\circ$ to 93.7\% at $90^\circ$ for audio eavesdropping on UGreen MAX 2 wireless headphone. In contrast, state-based inference prove remarkably resilient: the OIDIRE ODI-MF10A smart fan maintains a consistent 100\% ASR regardless of the orientation, while the Xiaomi 1S smart lamp exhibits sustains a high ASR with only minor fluctuations between $135^\circ$ and $225^\circ$. 
% These results show that state-inference attacks are generally less constrained by spatial geometry than high-fidelity audio recovery.
% Recovery performance mirrors these SNR patterns. The UGreen Studio MAX 2 headphone varies significantly, with attack success rate ranging from 50.5\% ($0^\circ$) to 93.72\% ($90^\circ$). State devices remain resilient: the OIDIRE ODI-MF10A fan maintains 0\% error regardless of angle, and the Xiaomi MI Desk Lamp 1S only exhibits minor classification errors from $135^\circ$ to $225^\circ$, proving that state attacks are generally less constrained by geometry than audio eavesdropping.

\begin{figure}[t]
    \centering
    \includegraphics[width=\linewidth]{figures/5.3/impact_angle.pdf}
    \caption{\alias{}'s robustness against antenna angles. The results indicate optimal performance at 90$^\circ$.}
    \label{fig:impact_angle}
\end{figure}


\textbf{Impact of barrier material.}
To evaluate the feasibility of \alias{} attack in real-world non-line-of-sight (NLoS) environments, we interpose various structural materials (including glass, wood, and concrete) into the propagation path between the adversary and the victim device. As shown in~\cref{fig:impact_barrier}(b), the presence of different barrier materials has a minimal influence on the SNR. For glass and wood, the SNR attenuation across all devices is approximately 1–2 dB compared to the line-of-sight (LoS) scenario. In contrast, concrete obstacles induce a more pronounced attenuation: the headphone experiences a 5.8~dB drop, while the fan and lamp exhibit reductions of 2.2~dB and 2.8~dB, respectively.
The impact of obstacles on ASR mirrors their effect on the SNR remains remarkably resilient across all tested materials, as shown in~\cref{fig:impact_barrier}(c). For glass and wood, the ASR remained unchanged across all devices.Even with the introduction of concrete, the impact is marginal: the headphone ASR decreases slightly by approximately 3\%, while the smart fan maintains a 100\% success rate. These results demonstrate that \alias{} is highly effective in penetrating common structural barriers, allowing covert through-wall eavesdropping in partitioned indoor environments such as offices and hotels. We demonstrate this with three case studies in~\cref{sec:case_studies}.

\begin{figure}[t]
    \centering
    \includegraphics[width=\linewidth]{figures/5.3/impact_barriers.pdf}
    \caption{Impact of structural barriers on \alias{} performance. The results demonstrate robust signal penetration through common building materials, enabling effective through-wall eavesdropping in real-world non-line-of-sight (NLoS) environments.}
    \label{fig:impact_barrier}
\end{figure}





%It is worth to mention that the difficulty of aligning the transmitting and receiving antennas with the target increases with distance. Although we attempted to align the antennas as precisely as possible during testing, optimal alignment could not be guaranteed. As shown in the results, the decline in both Attack Success Rate (ASR) and SNR accelerates as the distance increases.\hr{maybe not mention this}





