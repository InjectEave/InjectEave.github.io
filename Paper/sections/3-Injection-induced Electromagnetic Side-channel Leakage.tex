\section{Injection-Induced EM Side Channel}

Based on the observations and analysis above, this section provides the threat model and formulation of injection-induced EM side channels, and characterizes its feasibility in widely found nonlinear hardware within computer systems. 


\subsection{Threat Model}

We hypothesize that EM signals injected into nonlinear computer hardware can mix with and thus piggyback secret information within the target system; the modulated EM signals will then leak to side-channel adversaries. In particular, the injected EM energy boosts the amount of side-channel leakage, enabling adversaries to get information previously inaccessible with conventional EM side-channel methods.

\textbf{Adversary's Objective.} The objective of the adversary is the same as that in conventional EM side-channel analysis~\cite{kuhn2005electromagnetic, vuagnoux2009compromising, long2024eye}: inferring confidential information about a computer system's operations by analyzing the EM signals the adversary can collect. Our analysis in this work focuses on secrets in the form of analog signals, particularly the low-frequency signals such as human speech audio (below 20~kHz), power consumption signals (below 200~Hz), etc., which have been proving to be highly challenging targets [cite] due to the spectral mismatch between these secrets' frequencies and efficient EM leakage frequencies (MHz or GHz range). 

\textbf{Adversary's Capability.} We assume the adversary has a set of readily available commercial equipment that is able to both send EM injection signals and receive modulated EM emissions.
The added EM injection capability is the only difference from the assumed capability of conventional EM side-channel adversaries. The equipment often includes antennas, RF sources such as software-defined radio devices (e.g., USRPs [cite]), and potentially more advanced spectrum analyzers and signal generators commonly found in RF research labs. Similar to conventional EM side-channel research [cite many], we assume the adversary has prior knowledge of the target device's model and can acquire a similar device for profiling its effective EM injection and emission frequencies. 

\subsection{Leakage Modeling}
\label{sec:leakage_model}

\begin{figure}
    \centering
    \includegraphics[width=\linewidth]{figures/3.1/Leakage Model.pdf}
    \caption{Illustration of leakage modeling}
    \label{fig:placeholder}
\end{figure}


We formalize the Injection-Induced Side Channel through a \textit{Injection-Modulation-Emission} model. 

\textbf{Injection Coupling.} 
% Let $V_{sec}(t)$ denote the confidential internal analog signal (e.g., audio or sensor data) with a spectral bandwidth limited to baseband frequencies $f \ll f_c$. 
The adversary generates an injection carrier $V_{inj}(t)$ at frequency $f_c$: $V_{inj}(t) = A_{inj} \cos(2\pi f_c t)$. 
The target device's electrical traces act as unintentional receiving antennas, where the injection coupling efficiency is governed by a frequency-dependent injection transfer function $H_{rx}(f)$ (equivalently, $h_{rx}(t)$ in the time domain). The induced voltage $V_{c}(t)$ at the input of the vulnerable nonlinear component is:
\begin{equation}
    V_{c}(t) = |H_{rx}(f_c)| \cdot A_{inj} \cos(2\pi f_c t + \phi)
\end{equation}
Consequently, the total signal $V_{in}(t)$ present at the input terminal of the non-linear component is the superposition of the original secret signal and the coupled carrier:
\begin{equation}\label{eq:Vin}
    V_{in}(t) = V_{sec}(t) + V_{c}(t)
\end{equation}

\textbf{Non-Linear Modulation}
Following the series expansion model for semiconductor non-linearity established in prior EMI research\cite{kune2013ghost}, we approximate the transfer function of the hardware component as:
\begin{equation} \label{eq:serial}
    V_{out}(t) = \sum_{k=0}^{\infty} \alpha_k V_{in}^k(t) = \alpha_0 + \alpha_1 V_{in}(t) + \alpha_2 V_{in}^2(t) + \dots
\end{equation}
where $\alpha_k$ represents the $k$-th order coefficient. While the linear term $\alpha_1$ represents intended signal conditioning such as amplification, the quadratic term $\alpha_2$ and higher-order terms induce inter-modulation. Taking the quadratic term for example, substituting Equation~\ref{eq:Vin} into the quadratic term yields:
\begin{equation}
\label{eq:expansion}
\begin{split}
    \alpha_2 V_{in}^2(t) &= \alpha_2 [V_{sec}(t) + V_{c}(t)]^2 \\
    &= \alpha_2 [V_{sec}^2(t) + V_{c}^2(t) + \underbrace{2 V_{sec}(t) V_{c}(t)}_{\text{AM Modulation}}]
\end{split}
\end{equation}
The cross-product term in Equation~\ref{eq:expansion} exemplifies how the secret could be modulated onto the injected carrier. Denoting the total of signals carrying the modulated secret as $V^{c}_{sec}(t)$, the model shows: 
\begin{equation} \label{eq:Vmod}
    V^{c}_{sec}(t) = 2 \alpha_2 V_{sec}(t)V_{c}(t) + V^{c}_{hi}(t),
\end{equation}
where $V^{c}_{hi}(t)$ represents the high-order inter-modulation products associated with $\alpha_3, \alpha_4$, etc. 
This process effectively up-converts the spectral energy of $V_{sec}(t)$ from the baseband to the sidebands centered at $f_c$. 

\begin{figure}[t]
    \centering
    \includegraphics[width=\linewidth]{figures/3.2/Feasibility setup.pdf}
    \caption{Feasibility test setup.}
    \label{fig:feasibility test setup}
\end{figure}

\begin{figure*}[th!]
    \centering
    \includegraphics[width=1.0\linewidth]{figures/3.3/feasibility.pdf}
    \caption{Feasibility test.}
    \label{fig:feasibility test.}
\end{figure*}

% \begin{figure}[!th]
%     \centering
%     \includegraphics[width=0.9\linewidth]{figures/3.3/RES_Powerline.pdf}
%     \caption{Feasibility test on power converter \ly{combine this figure with fig 5, so 4 components displayed in one figure}}
%     \label{fig:feasibility_power}
% \end{figure}

\textbf{Carrier Emission}
The modulated signal $V^{c}_{sec}(t)$ propagates through the device's conductive paths and emits at other electrical interconnects which act as unintentional transmitting antennas. The leakage efficiency is determined by the emission transfer function $H_{tx}(f)$. The final leakage signal $V_{eav}(t)$ observed by the adversary is:
\begin{equation} \label{eq:Veav}
    V_{eav}(t) = 2 \alpha_2\cdot h_{tx}( V_{sec}(t)V_{c}(t)) + h_{tx}(V^{c}_{hi}(t))
\end{equation}
When only considering the dominant second-order inter-modulation for simplicity, the leakage amplitude $|V_{eav}(t)|$ can be expressed as a function of the system parameters:
\begin{equation} \label{eq:effi}
   |V_{eav}(t)| \propto \underbrace{|H_{tx}(f_c) H_{rx}(f_c)|}_{\text{Coupling Efficiency}} \cdot \underbrace{|\alpha_2|}_{\text{Non-Linearity}} \cdot \underbrace{|V_{sec}(t)|\cdot|V_{inj}(t)|}_{\text{Signal Amplitude}}
\end{equation}





The modeling reveals that even if the original secret signals' amplitudes are low and their frequencies are significantly lower than the efficient emission frequencies of $H_{tx}(f)$, the adversary can amplify the leakage by tuning the injection frequency $f_c$ to maximize the compound efficiency product $|H_{tx}(f_c) H_{rx}(f_c)|$, as well as by increasing the amplitude $A_{inj}$ of EM injection $V_{inj}(t)$. 










\subsection{Feasibility Analysis}~\label{sec:feasibility}
To verify this hypothesized injection-induced leakage caused by inter-modulations of $V_{sec}(t)$ and $V_{inj}(t)$, we individually characterized the leakage behavior of four types of the most common nonlinear hardware found in computer systems, including (1) amplifiers, (2) analog-to-digital converters (ADCs), (3) Switching Metal-Oxide-Semiconductor Field-Effect Transistor(MOSFET), and (4) Power converters. As shown by the setup in Figure~\ref{fig:feasibility test setup}, we used two near-field electromagnetic probes as transmitting and receiving antennas to inject EM energy into and receive emissions from the nonlinear components. 


\subsubsection{Susceptibility of Common Nonlinear Electronics}

For each nonlinear component, we measured the electromagnetic emissions under both conventional ``passive'' eavesdropping conditions and our proposed ``active injection'' conditions. This comparative analysis aims to illustrate the leakage enhancement capability provided by the injection-induced side channels. In addition, we also measured and obsevered no leakage signals when the nonlinear components were replaced by a linear resistance, confirming that hardware nonlinearity is the key enabling such injection-induced leakage. 

\textbf{(1) Amplifier.} Amplifiers are well-known nonlinear devices~\cite{kune2013ghost, tu2019trick} often found in data interfaces such as sensor input and actuator output circuits. 
We selected the AD623 [cite], a widely used rail-to-rail instrumentation amplifier, as our primary target to validate the model in Section~\ref{sec:leakage_model}. To characterize its non-linear response in a controlled setting, we used a signal generator to directly input a baseband frequency sweep signal, denoted as $V_{sec}(t)$, into the amplifier's input terminals. The sweep secret signal ranged from 0 to 16~kHz with an amplitude of 200~mV, simulating a robust analog sensor input. Simultaneously, we targeted the device with an EM injection carrier $V_{inj}(t)$ at 80~MHz.

The results, visualized in \cref{fig:feasibility test.}, clearly shows the spectral content of the recovered leakage signal. However, we also observed distinct harmonic components ($2\times f_{secret}$, $3\times f_{secret}$, ...)\ly{need to verify and correct the multiples} in the demodulated spectrum in addition to the fundamental frequency of the input secret signals. 
This observation further provides proof of the hardware's non-linearity, revealing the impact of $V^{c}_{hi}(t)$ in Equation~\ref{eq:Vmod} in distorting the original secret information. In contrast, passive eavesdropping of the 0--16kHz analog secret shows no visible signals on the receiver end.






\textbf{(2) ADC.}
% \begin{figure}
%     \centering
%     \includegraphics[width=\linewidth]{figures/3.3/feasibility_adc.pdf}
%     \caption{Feasibility test on ADC}
%     \label{fig:feasibility_ADC}
% \end{figure}
Analog-to-digital converters (ADCs), another type of ubiquitous component in modern digital computer systems that process inputs of analog physical information, also has nonlinear characteristics.  
We used ADS1115 [cite], a common 16-bit ADC, as the test target. Similar to the amplifier test, we input a baseband sweep signal $V_{sec}(t)$ while targeting the device with the RF carrier at 80MHz. 

As shown in \cref{fig:feasibility test.}, clear secrets and their harmonic signals are observed in the injection-induced leakage, while the conventional side channel analysis receives no useful information. Notably, the results (Fig. 5d) reveal that leakage persists even when the injection bandwidth exceeds far beyong the ADC's sample rate, which is 860 Hz, revealing that modulation occurs in the continuous-time analog front-end.



\textbf{(3) MOSFET in Actuator Driver.}
% \begin{figure}
%     \centering
%     \includegraphics[width=\linewidth]{figures/3.3/feasibility_gpio.pdf}
%     \caption{Feasibility test on Switching MOSFET}
%     \label{fig:feasibility_MOSFET}
% \end{figure}
% MOSFETs are used in switching circuits, such as GPIOs, to... \ly{need to add a description of its usage}
% We built a discrete MOSFET circuit driving a resistive load to evaluate leakage in switching electronics. An Arduino controller generated a frequency-stepped square wave signal $V_{sec}(t)$ shifting between 300 Hz, 600 Hz, 900 Hz, and 1200 Hz, connected to the Gate \ly{what is the gate? Confusing. Try to clarify the setup better}.
\ly{Todo: need to replace the name. Actuator driver is not a known term} Actuator drivers are the essential physical interfaces that allow computing systems to control external hardware, such as motors in smart fans, heating elements, or relays in smart plugs. These circuits typically employ power MOSFETs acting as high-speed switches to modulate power delivery via techniques like Pulse Width Modulation (PWM).  \ly{You haven't mentioned what part is nonlinear in this ``actuator driver''} To evaluate its injection-induced leakage, we constructed a representative driver circuit using a discrete MOSFET driving a resistive load. An Arduino controller was used to generate the baseband secret signal, $V_{sec}(t)$, in the form of a frequency-stepped square wave (shifting between 300 Hz, 600 Hz, 900 Hz, and 1200 Hz). This signal simulates a typical variable-speed motor control sequence. Simultaneously, the EM injection carrier $V_{inj}(t)$ couples onto the high-current loop formed by the Drain-Source path and the load.

The recovered spectrogram shown in Figure~\ref{fig:feasibility test.} displays a clear "staircase" pattern corresponding to the frequency steps. Still, strong harmonics accompany each fundamental frequency.


\textbf{(4) Power Converter.}
The AC-DC rectification stage is the power entry point for almost all modern electronics, responsible for converting high-voltage AC mains into DC power. 
To investigate leakage from power supply units, we connected a power converter with a fixed high-power resistive load. Here, the secret signal could be the AC mains voltage itself ($V_{sec}(t)$ at 50 Hz). We targeted the AC power cable with the injection carrier.



The measurement results (\cref{fig:feasibility test.}) reveal a prominent modulation effect: the fundamental mains frequency and a rich set of harmonics (e.g., 100 Hz, 150 Hz) are clearly recovered as sidebands around the carrier. 
This strong modulation arises from the bridge rectifier diodes at the adapter's input. Again, no signal appeared in the passively eavesdropped traces. 


 


\subsubsection{Leakage Characteristics}

After confirming the existence of injection-induced leakage, we further seek to characterize the quantitative relationships between the injection, leakage, and secret signals. We observe that injection-induced side channels produce both the benefit of secret signal amplification and the challenge of nonlinear distortions. 

\textbf{Benefit of Linear Amplification.}
We reused the setup above to produce different strengths of secret and injection signals to examine the quantitative relationship revealed by Equation~\ref{eq:effi}. 
Our experiments confirm the near-linear relationship between $|V_{eav}(t)|$, $|V_{sec}(t)|$, and $|V_{inj}(t)|$. For example, Figure \ref{fig:lamp_power_sideband} shows the variations of these quantities on the AD623 amplifier, where $|V_{eav}(t)|$ scales almost proportionally with $|V_{sec}(t)|$ and $|V_{sec}(t)|$. Results of the other three nonlinear components exhibit highly similar trends and are thus omitted. This result thus demonstrates the capability of injection-induced EM side channels in performing fine-grained signal analysis, where the eavesdropped signals can vary continuously according to the original analog secrets and the intended amplification controlled by the EM injection signal. 

\begin{figure}[t]
    \centering
    \includegraphics[width=1\linewidth]{figures/3.2/RES_LAMP_SECRET_TX.pdf}
    \caption{Relationship between injection/secret signal power and the resulting sideband amplitude.}
    \label{fig:lamp_power_sideband}
\end{figure}

\textbf{Challenge of Nonlinear Distortions.}
Although the linear amplification effect of the second-term inter-modulation provides injection-induced side channels with the unique capability of controllable leakage strength, it also inevitably faces the distortions caused by the higher-order terms ($V^{c}_{hi}(t)$ in Equation~\ref{eq:Vmod}). This is illustrated by the nonlinear variations of the data points in Figure~\ref{fig:lamp_power_sideband}, and could manifest as harmonics of the original secret signal  (e.g., as shown in Figure~\ref{fig:feasibility test.}). While the harmonics are discernible when the secret is a simple single tone, they may degrade the quality of more complex wideband signals by contaminating the original frequency components. In particular, human speech signals naturally consist of a fundamental component and its harmonics, meaning that higher-order intermodulation products of the fundamental and even lower-order harmonics can overlap with, and thereby distort, higher-frequency components, as shown in Figure~\ly{Need to add a figure to show this}, posing a unique challenge for reconstructing high-quality audio and other wideband secrets in injection-induced side channels.   






