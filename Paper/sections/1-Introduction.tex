\section{Introduction}
This work investigates the new problem statement of exploiting active electromagnetic injection and the inherent nonlinearity of computer hardware to reshape the capability of electromagnetic side-channel analysis. Side-channel analysis has become one of the most important types of security analysis methodologies, exploiting the non-ideal abstractions of computer systems to compromise confidentiality and achieve unauthorized access to data internal to a protected device~\cite{agrawal2002side,standaert2009introduction}. By collecting and analyzing signals unintentionally produced by the physical operations of computer hardware, such as sound~\cite{al2016acoustic, backes2010acoustic}, light~\cite{ferrigno2008aes,loughry2002information}, and electromagnetic emissions~\cite{kuhn2002optical, long2024eye, vuagnoux2009compromising}, adversaries would be able to infer critical information of cryptographic operations~\cite{gnad2019leaky,camurati2018screaming,genkin2015stealing}, confidential input data~\cite{jin2021periscope, halevi2015keyboard, bolton2023characterizing}, and private user biometrics~\cite{long2024eye,li2025emiris,ni2023recovering}.


\begin{figure}
    \centering
    \includegraphics[width=1\linewidth]{figures/Overview.pdf}
    \caption{Overview figure TBA}
    \label{fig:placeholder}
\end{figure}

Among various methods of side channels, electromagnetic (EM) side channels present a highly pervasive and impactful attack surface, as all modern computer systems rely on current and voltage variations in electrical circuits to perform basic computation and communication jobs. The resulting time-varying EM fields inevitably radiate into the surrounding environment and propagate through the air to nearby adversaries. For example, the TEMPEST project in the 1980s~\cite{van1985electromagnetic}... Since then, the security community has shown the feasibility of exploiting EM side-channel leakage to eavesdrop on a wide range of secret information, such as user inputs on keyboards~\cite{vuagnoux2009compromising} and touchscreens~\cite{jin2021periscope}, audios~\cite{choi2020tempest,chen2024eavesdropping}, smartphone screens~\cite{liu2020screen}, users' biometric data~\cite{li2025emiris,ni2023recovering,xu2025empalm}, and even confidential video streams of smart home cameras~\cite{long2024eye}.  

Despite the massive theoretical attack surface, existing EM side channel research has revealed a critical limitation in the range of applicable eavesdropping distances and observable types of information. Specifically, the EM energy that can propagate to external adversaries is solely determined by the target's internal characteristics, including the amplitude of the current/voltage that carries the secret information, the frequency of the internal electrical signals, and the EM transfer efficiency of the target's circuits that act as unintentional radiating antennas. This major limitation is rooted in the threat model assumption that the side-channel analyzer can only passively observe the EM leakage of a target device. 
As a result, \textit{conventional side-channel eavesdroppers face the seemingly ``unsolvable'' problem of low signal-to-noise ratio (SNR)}, treating better EM receivers as their only measure for improving EM side-channel analysis capabilities. 

Toward overcoming this grand challenge, our work rethinks this core assumption of passive eavesdropping, and provides a new analytical framework that employs active EM injection to induce and amplify EM side-channel leakage in a controllable manner. Our key observation is that there often exists a mismatch between the frequencies of target secret signals and the efficient EM coupling frequencies of the target hardware circuits, which poses the fundamental limit on the amount of secrets' EM energy that can \textit{\textbf{leave}} the target. However, we also observe that existing research on active EM injection attacks~\cite{kune2013ghost,tu2018injected} has shown and modeled how external EM signals can be designed to be at the most efficient coupling frequencies to \textbf{\textit{enter}} the target device. Meanwhile, injected EM signals could be unintentionally demodulated by nonlinear hardware such as amplifiers, allowing adversaries to use EM carriers to inject even false \textbf{\textit{low-frequency analog signals}} into target systems. In addition, recent research~\cite{kaji2023echo, pu2025your} has also demonstrated an unintentional backscattering effect of some serial transmission interfaces that allows adversaries to eavesdrop on transmitted digital bits by sending an EM signal and analyzing its reflections. Nevertheless, the analysis was limited to coarse-grained impedance changes caused by high-voltage swings in digital data transmission, leaving the fundamental leakage mechanism and the susceptibility of analog secrets unexplored.   
Building upon these previous theories, we synthesize a new threat model of \textit{Injection-Induced EM Side Channel} to bridge the theoretical and experimental gaps between side channel leakage and EM injection. \textit{We hypothesize that 
nonlinear computer hardware can modulate secret information, in the form of electrical inputs of these hardware components, onto injected carriers, which will then emit and propagate to side-channel eavesdroppers.} 

If this hypothesis were true, then EM side-channel eavesdroppers would be able to actively control their injected EM carriers to break through the target devices' EM security boundaries and greatly extend the limits of recoverable information. We characterize this threat model with experiments on four types of the most typical nonlinear hardware components found in computer systems, including amplifiers, analog-to-digital converters, power converters, and switching MOSFETs. Our measurements verify that injection-induced side channels enable unconventional attack vectors, such as eavesdropping on secret information with EM frequencies on the order of 10~Hz--1~MHz, which itself could be too low to propagate to external eavesdroppers. 
Our theoretical modeling further quantifies the relationship between the injected EM carrier, secret signals, and recoverable information, providing a framework for analyzing threats against the most common analog data interfaces, such as audio output and input, control signals of actuators, and even device power traces. 

Despite these new capabilities, our tests show that secret signals eavesdropped with this approach unavoidably suffer from higher-order inter-modulations that add harmonics of an original signal to its spectrum. This nonlinearity-specific distortion poses unique challenges to eavesdroppers who aim to recover wide-band signals, such as human speech audio, whose lower-frequency components' harmonics overlap with and thus contaminate higher-frequency information. To recover higher-fidelity secrets, we 
utilize our developed quantitative model of the injection-induced side channel leakage process to simulate distorted received signals with different profiles of environmental noise and build a diffusion-based denoising model conditioned on the simulated data. Tests on physically collected speech audio data show the improvements of 25\%--230\% on different audio quality metrics \ly{update the numbers}. 

Our evaluation in lab settings first identifies typical low-frequency analog secrets carried by 15 commercial off-of-the-shelf (COTS) household devices. Audio of headphones, landline phones, and microphone sensor inputs could leak both speaker identity and spoken content. Controlling signals and power consumption traces of IoT devices, such as smart fans and lamps, could leak personal activities in households. Our tests show that injection-induced side channel attacks could eavesdrop on the majority of these devices from over 2 m away and through walls, with a maximum distance of 6 m for recovering intelligible headphone audio. 
Our case studies of audio eavesdropping in several personal, public, and work scenarios further demonstrate its security consequences in the wild. An example of landline phones showcases the new capability of integrating EM injection and injection-induced leakage to achieve closed-loop, context-aware eavesdropping and manipulation on private conversations. Finally, we discuss the other possible analog and digital secrets that need to be further threat-modeled in future research, and analyze the possible mitigation and paths toward systematic protections against integrated EM injection and side channel attacks. The main contributions of this work is summarized as follows:
\begin{itemize}
    \item Theoretical framework for Injection-Induced EM Side Channels. Our modeling and experiments identify ubiquitous nonlinear hardware in computer systems as the root cause of this new phenomenon, enabling a wide range of future EM security research integrating closed-loop data eavesdropping and manipulation analysis.  
    \item Technical designs of eavesdropping. Our hardware and algorithm designs exemplify how to exploit this emerging vulnerability, showing the unprecedented capability of  
    eavesdropping on low-frequency analog secrets from non-trivial distances through EM side channels. 
    \item Characterization of threats and mitigations. Our evaluation on 15 COTS devices gauges the new threat model's impact on typical audio-processing and smart home devices, based on which we analyze possible defense methodologies. 
\end{itemize}


