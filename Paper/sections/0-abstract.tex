\begin{abstract}
Electromagnetic (EM) side-channel leakage and injection are typically treated as distinct physical phenomena, threatening data confidentiality and integrity respectively. This work investigates how EM injection can be used to amplify side-channel leakage that is otherwise infeasible. We introduce a novel framework for \textit{Injection-Induced EM Side Channels} to enable integrated, closed-loop EM security analysis. Our theoretical modeling and experimental measurements reveal that nonlinear hardware components, such as ubiquitous amplifiers, analog-to-digital converters, and power converters, can modulate secret analog signals onto an injected EM carrier and thus upconvert low-frequency secrets into measurable EM emissions. By tuning the injection frequency and amplitude, adversaries gain the ability to actively shape the effective spectrum and entropy of the resulting leakage. We demonstrate eavesdropping on wired and wireless headphones' audio from up to 6 m away, as well as in through-wall scenarios, and characterize injection-induced EM leakage of other low-frequency secrets such as power consumption of smart home devices and analog sensor inputs. Case studies further demonstrate how the proposed techniques enable closed-loop eavesdropping and manipulation of landline phones' conversations. Finally, we analyze the broader security challenges posed by this emerging threat vector and outline directions for systematic threat characterization and protection.
\end{abstract}